\section{Introduction}
% General comments
% Note that these are still philosophical
% You can do lit review here (can have subheaders)
%   - 

% Ambigtious goal: Build flexible and scalable physical intelligence for planning and understanding
% Talk about the nature of physics and its role in AI problems
%   There is general stuff and there is specific stuff
%   What we'd like to do is to
%       - perceive a scene and construct useful intermediate, composable, and resuable representations in the temporal and spatial hierarchy that we can use to plan and understand
%       - Intuitive Physics: (Here talk about why intuitive physics is important)
%           - object concepts
%           - read other literature on that here
%           - 
%       - Why do we want physical intelligence? It is because it can ACCELERATE LEARNING for other tasks, that's why.
%       - A big problem is how to GET to the intermediate representation
%       - But another problem is how do you manipulate a higher level representation for actual planning and simulation and understanding
%   - Two types of physical scene understanding: static and dynamic: hmmm, perhaps you can literature review on this?


%   - Ingredients for the above/criteria for PHYSICAL intellgence:
%       - 1. Build causal models of the world that support explanation and understanding (causal INTERPRETATION)
%           --> why learning a simulator is the way to go
%           -- note that intuitive physics is causal
%           -- "Simulation-based models can also capture how people make hypothetical or counterfactual predictions"
%       - 2. harness compositionality [and learning-to-learn] to rapidly acquire and generalize knowledge to new tasks and situations.
%           --> Talk about HOW compositionality allows you to generalize knowledge to new tasks and situations and WHY that is required for the ambitious goal
%       -  Cognition and thought is about using these models to understand the world, explain what we see, imagine what could have happened that didn’t, what could be true that isn’t, and then planning actions to make it so. The difference between pattern recognition and model-building, between prediction and explanation, is central to our view of human intelligence.
%           - My model is doing prediction. Does it do any explaining? What does explaining mean?
%               - Well one example of strong explanation is inferring the EXISTENCE of an unknown latent property (global parameters)
%               - A weaker form of explanation is to inferring the IDENTITY/NATURE of a KNOWN latent property (local parameters)
%           - A simulator can ``imagine what could have happened that didn’t, what could be true that isn’t''
%           - the process of building that simulator is the process of MODEL BUILDING
%       - Let this discussion of model building be your transition to "Two General Approaches"
%   - Building Models = Building Theories? --> Need to read more on this to see if this is necessary


% Two General Approaches: 
%   Reminder: we want scalability, flexibility, accuracy, and efficiency
%   Discuss both approaches
%       Both are forms of program induction: learning is a search for the programs likely to have generated the data. The space of programs are different and how the programs are found is different. They are both trying to learn the program.
%       Introduce IPE: A promising recent approach sees intuitive physical reasoning as similar to inference over a physics software engine, the kind of simulators that power modern-day animations and games
%       Introduce RNN: Temporal Restricted Boltzmann Machine
%       model building vs pattern recognition
%       IPE vs RNN pros and cons of each
%       IPE con
%           hard to automate
%               Read: Exploiting compositionality to explore a large space of model structures.
%           Relative to physical ground truth, the intuitive physical state representation is approximate and probabilistic, and oversimplified and incomplete in many ways.
%       How does IPE and RNN view the object of computation?
%           IPE: TODO
%           RNN: the "objects" hopefully will emerge 
%       in light of the pros and cons summarize that we want pros of both
%   My model draws on the strengths of both: model building as general structure, but that model building only works because we use patern recognition to fill in the specific stuff

% Sketch Paradigm
%   My framework is doing a form of model building/explanation. The specific way it does it (it's just the baby step towards the right way to do it) is to start with a general sketch and then learn the specifics.
%   "A primary job of learning is to extend and enrich these [causal] models [of the world]", which seeks to explain data of what we see. So my model is not LEARNING to find the global parameters/causal model itself, but the general framework is given, and my model fills in/extends/enriches the sketch to EXPLAIN the SPECIFICS of what we see UNDER the FRAMEWORK of the origianl general sketch.

% Why my model is important, how it links to all the ambitious goal, the general approaches, and the sketch paradigm, and what has this linkage given the model the ability to do
%   - compositionality: decomposes the problem --> useful for dealing with complexity

% Outline of the paper





% paragraph on scene understanding through simulation? Or Do we talk about scene understanding and then say that simulation is a paritcular way to get there or (THE) way to get tehre?
[Paragraph on physical scene understanding. Give examples of the REALIZATION of it in humans and list what we can do with that ability and why it matters. Give examples on the NEED for it in AI systems, building off of the humans.]
% 

% OR, you can give examples on the NEED for AI systems and use the REALIZATION of it in humans to support that
% Consolidate NEEDS from papers that you read
%   -
%   -
%   -

% Intro
% why physical scene understanding is important for intelligence
%   intuitive physics
%   implications on robotics, games, etc
%   talk about what we desire: accuracy, efficiency, flexible, scalable (generalization is part of this)
%       explain why each of these is desirable with an example

% Here you can cite some works and successes... One approach (A et al, B et al) does this and the other approach (C et al, D et al) does that
Two general approaches have recently emerged to [address this problem] % Both are about learning a PROGRAM/Building a MODEL/program
% two ways to approach it and their scalability issues. "Both are in some form flexible and scalable
%   use some sort of physics engine which has strong priors on the constraints
%   RNN/MLP
% have complementary pros and cons
In contrast to both of these approaches, our work []. [Compare with phys engine and say how it addresses advantage/disadvantage]. [Compare with rnn and say how it addresses advantage/disadvantage]. [Summarize the advtg to both: "It has the advantage of IPE and the advantage of RNN (at the cost of disadvantage of IPE and disadvantage of RNN; actually we shouldn't say the RNN because the reason why are doing this work is because it is better than both approaches.)] we are trying to combine the strengths of both

% when you contrast with phys engine, you can talk about the nature and limitations of physics engines and how it is difficult to automate the code, both in the simulation (collision solving) and adapatability of the world definition (can it LEARN to handle new worlds without manual specification/coding?)

% 

% sketch pardigm (from Kevin Ellis's professor)
%   learning as a form of model building
%   general —> specific
% somewhere in there talk about compositionally, causality, generalizability, and what we desire in 
% connect how your work is a baby step toward the ambitious goal of what we desire

% Summarizing sentence here: "our work attempts to combein the best of both and follows the sketch paradigm" (So actually everything you said above was just philosophical discussion and arguments for why your approach works) 
% By now the philosophical niceties of your approach have been discussed. Now let's get more specific on what our model actually does. 
% Description (specifics)
%   Compositionality
%   Causality
% Abilities
%   Generalization: say how it is BECAUSE OF THIS compositional/causality structure that allows this success generalization (may want to note that TRUE generalization: generalization across objects, number of objects, properties, few examples, etc is hard to do (you mshould have mentioned this in your description of the IPE. But state here that your model takes the crucial prelimiarny baby step towards that ambitious goal. So it si good that you put your generalization capabilities in the context of the bigger picture of what we want in generalization.
%    Because it is a DIFFERENTIABLE and it is a SIMULATOR: we can query it to answer inference questions, like what the input MUSt HAVE BEEN to generate this outcome.
% THis is about the key results of your model
%   Also that our model performs orders of magnitude better than vanilla RNNS and that the pairwise compositionality also is a better architecture




% * Intuitive Physics \\
% * Learning as Model Building \\
% -- Builds model of dynamics through learning a simulator; and from this simulator/model we can accomplish other tasks, such as inferring mass, size, shape \\
% * Simulation  Scene Understanding \\
% -- Learning to simulate = learning to understand/reason \\
% -- Learning a time-evolution Operator \\
% * Because IPEs approximate game engines, and I am learning to simulate a game engine, this model learns to simulate an intuitive physics engine in a sense \\
% * Physics Engines are expensive and inflexible and approximate
% -- talk about motivation about learning physics from observation \\
% \textbf{Things to emphasize?} \\
% * Framework? \\
% --Causality: causality is given in the structure of the model, but it is not learned \\
% --Harness Compositionality \\
% --Work at the high level representation \\
% --Use baselines to illustrate “how much prior information you should incorporate into your model”  \\
% --> Structural biases improve prediction and generalization. \\
% * Model Architecture? \\
% -- Neighborhood \\
% * Dataset? \\
% * Strong Generalization over number of objects \\
% * Model Building \\
% -- Shows that the NPE discovers/explains the role of factors such as mass and shape and size \\
% -- Amortized inference of future state given past state